\documentclass[paper=letter,fontsize=11pt]{scrartcl} % KOMA-article class

\usepackage[slovak]{babel}
\usepackage[T1]{fontenc}
\usepackage[utf8x]{inputenc}
\usepackage[protrusion=true,expansion=true]{microtype}
\usepackage{amsmath,amsfonts,amsthm}     % Math packages
\usepackage{graphicx}                    % Enable pdflatex
\usepackage[svgnames]{xcolor}            % Colors by their 'svgnames'
\usepackage{geometry}
	%\textheight=700px                    % Saving trees ;-)
%\usepackage{url}
\usepackage[colorlinks=true,
linkcolor=blue,
urlcolor=blue]{hyperref}
\usepackage{float}
\usepackage{etaremune}
\usepackage{wrapfig}

\usepackage{attachfile}

\frenchspacing              % Better looking spacings after periods
\pagestyle{empty}           % No pagenumbers/headers/footers

%\addtolength{\voffset}{-40pt}
%\addtolength{\textheight}{20pt}

\setlength\topmargin{0pt}
\addtolength\topmargin{-\headheight}
\addtolength\topmargin{-\headsep}
\setlength\oddsidemargin{0pt}
\setlength\textwidth{\paperwidth}
\addtolength\textwidth{-2in}
\setlength\textheight{\paperheight}
%\addtolength\textheight{-3in}
\addtolength\textheight{-2in}
\usepackage{layout}

%%% Custom sectioning}{sectsty package)
%%% ------------------------------------------------------------
\usepackage{sectsty}

\sectionfont{%			            % Change font of \section command
	\usefont{OT1}{phv}{b}{n}%		% bch-b-n: CharterBT-Bold font
	\sectionrule{0pt}{0pt}{-5pt}{1pt}}

%%% Macros
%%% ------------------------------------------------------------
\newlength{\spacebox}
\settowidth{\spacebox}{8888888888}			% Box to align text
\newcommand{\sepspace}{\vspace*{1em}}		% Vertical space macro

\newcommand{\MyName}[1]{ % Name
		\Huge \usefont{OT1}{phv}{b}{n} \hfill #1
		\par \normalsize \normalfont}
		
\newcommand{\MySlogan}[1]{ % Slogan}{optional)
		\large \usefont{OT1}{phv}{m}{n}\hfill \textit{#1}
		\par \normalsize \normalfont}

\newcommand{\NewPart}[2]{\section*{\uppercase{#1} \small \normalfont #2}}

\newcommand{\NewParttwo}[1]{
		\noindent \huge \textbf{#1}
        \normalsize \par}



\newcommand{\PersonalEntry}[2]{\small
		\noindent\hangindent=2em\hangafter=2 % Indentation
		\parbox{10em}{        % Box to align text
		\textit{#1}}		       % Entry name}{birth, address, etc.)
		\small\hspace{1.5em} #2 \par}    % Entry value

\newcommand{\SkillsEntry}[2]{      % Same as \PersonalEntry
		\noindent\hangindent=2em\hangafter=0 % Indentation
		\parbox{\spacebox}{        % Box to align text
		\textit{#1}}			   % Entry name}{birth, address, etc.)
		\hspace{1.5em} #2 \par}    % Entry value	
		
\newcommand{\EducationEntry}[4]{
		\noindent \textbf{#1} \hfill      % Study
		\colorbox{White}{%
			\parbox{8em}{%
			\hfill\color{Black}#2}} \par  % Duration
		\noindent \textit{#3} \par        % School
		\noindent\hangindent=2em\hangafter=0 \small #4 % Description
		\normalsize \par}

\newcommand{\WorkEntry}[5]{
		\noindent \textbf{#1}
        \noindent \small \textit{#2}
        \hfill      % Study
        \colorbox{White}{%
			\parbox{6em}{%
			\hfill\color{Black}#3}} \par  % Duration
		\noindent \textit{#4} \par        % School
		\noindent\hangindent=2em\hangafter=0 \small #5 % Description
		\normalsize \par}

\newcommand{\Language}[2]{
		\noindent \textbf{#1}
        \noindent \small \textit{#2}}
        
\newcommand{\Text}[1]{\par       
		\noindent \small #1 
		\normalsize \par}
        
\newcommand{\Textlong}[4]{
		\noindent \textbf{#1} \par
        \sepspace
        \noindent \small #2
        \par\sepspace      
		\noindent \small #3
        \par\sepspace      
		\noindent \small #4
        \normalsize \par}

\newcommand{\PaperEntry}[7]{
		\noindent #1, ``\href{#7}{#2}", \textit{#3} \textbf{#4}, #5 (#6).}

\newcommand{\ArxivEntry}[3]{
		\noindent #1, ``\href{http://arxiv.org/abs/#3}{#2}", \textit{{cond-mat/}#3}.}
        
\newcommand{\BookEntry}[4]{
		\noindent #1, ``\href{#3}{#4}", \textit{#3}.}
        
\newcommand{\FundingEntry}[5]{
        \noindent #1, ``#2", \$#3 (#4, #5).}

\newcommand{\TalkEntry}[4]{
		\noindent #1, #2, #3 #4}

\newcommand{\ThesisEntry}[5]{
		\noindent #1 -- #2 #3 ``#4" \textit{#5}}

\newcommand{\CourseEntry}[3]{
		\noindent \item{#1: \textbf{#2} \\ #3}}

%%% Begin Document
%%% ------------------------------------------------------------
\begin{document}

\MyName{Denisa Rudincová}

%%% Personal details
%%% ------------------------------------------------------------
\NewPart{}{}

\PersonalEntry{Dátum narodenia}{28. 3. 1816}
\PersonalEntry{Adresa}{Fialková 42, 075 01 Lentilkovo, Slovensko}
\PersonalEntry{Telefón}{+421970515414}
\PersonalEntry{Mail}{\url{524996@mail.muni.cz}}
%%% Objective
%%% ------------------------------------------------------------

\NewPart{Ciele}{}

Mojím hlavným cieľom je rozvíjať svoje technické zručnosti a~programátorské schopnosti prostredníctvom praktických projektov a~výziev v~oblasti informatiky.

%%% Education
%%% ------------------------------------------------------------
\NewPart{Vzdelanie}{}

\EducationEntry{Informatika}{2021 – doteraz}{Fakulta informatiky\\Masarykova univerzita\\Brno, Česká republika}
 
\sepspace

\EducationEntry{Osemročné štúdium
}{2013 – 2021}{Gymnázium\\Bohviekde, Slovensko}

%%% Skills 
%%% ------------------------------------------------------------
\NewPart{Programátorské schopnosti}{}

\begin{itemize}
  \item \textbf{Najviac skúseností:} Python, C, Java, Perl, HTML
  \item \textbf{Nejaké skúsenosti:} C\#, HTML, CSS, Oracle SQL
  \item \textbf{Dotkla som sa:} PHP, Perl, Haskell, C++, 3D tlač
\end{itemize}

%%% Experience
%%% ------------------------------------------------------------
\NewPart{Pracovné skúsenosti}{}

\WorkEntry{Programátorka}{CVT FI MU}{}{júl 2023 – doteraz}{}
{Podieľanie sa na vývoji Informačného systému Masarykovej univerzity.}

\sepspace
\WorkEntry{Digital Portfolio Marketer}{localhost.company}{}{júl 2021 – december 2022}{}{Mojou úlohou bolo spravovať sociálne siete firmy, aktualizovať webové stránky a~dopĺňať ich portfólio.}

%%% Projects
%%% ------------------------------------------------------------
\NewPart{Projekty}{}

\begin{itemize}
  \item \textbf{Knižnica deskriptorov pre mikroorganizmy}, projekt z~predmetu PV162.

\end{itemize}

\end{document}